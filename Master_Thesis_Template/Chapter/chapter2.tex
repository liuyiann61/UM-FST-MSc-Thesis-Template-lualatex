\chapter{\MakeUppercase{[Type Chapter Title]}}
[Type Chapter]
\section{\MakeUppercase{[Type Chapter Section Title]}}
[Type Chapter Section]
\subsection{\textsc{[Type Chapter Subsection Title]}}
[Type Chapter Subsection]
\subsubsection{[Type Chapter Subsubsection Title]}
[Type Chapter Subsubsection]
\subsubsection{Figure}
Figure \ref{fig:2:1}, Figure \ref{fig:2:2a} and Figure \ref{fig:2:2}(b)--(c) show the same example.

\begin{figure}[!htb]
	\centering
	\includegraphics[width=0.9\linewidth]{Figures/Example.pdf}
	\caption{Example of a figure.}
	\label{fig:2:1}
\end{figure}

\begin{figure}[!htb]
	\centering
	\begin{subfigure}[b]{0.45\linewidth}
		\includegraphics[width=\linewidth]{Figures/Example.pdf}
		\caption{}
		\label{fig:2:2a}
	\end{subfigure}
	\hfil
	\begin{subfigure}[b]{0.45\linewidth}
		\includegraphics[width=\linewidth]{Figures/Example.pdf}
		\caption{}
		\label{fig:2:2b}
	\end{subfigure}
	\hfil
	\begin{subfigure}[b]{0.45\linewidth}
		\includegraphics[width=\linewidth]{Figures/Example.pdf}
		\caption{}
		\label{fig:2:2c}
	\end{subfigure}
	\hfil
	\begin{subfigure}[b]{0.45\linewidth}
		\includegraphics[width=\linewidth]{Figures/Example.pdf}
		\caption{}
		\label{fig:2:2d}
	\end{subfigure}
	\caption{Example of multi-figure.}
	\label{fig:2:2}
\end{figure}